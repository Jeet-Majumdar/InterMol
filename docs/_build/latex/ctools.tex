% Generated by Sphinx.
\def\sphinxdocclass{report}
\documentclass[letterpaper,10pt,english]{sphinxmanual}
\usepackage[utf8]{inputenc}
\DeclareUnicodeCharacter{00A0}{\nobreakspace}
\usepackage[T1]{fontenc}
\usepackage{babel}
\usepackage{times}
\usepackage[Bjarne]{fncychap}
\usepackage{longtable}
\usepackage{sphinx}


\title{ctools Documentation}
\date{July 05, 2011}
\release{a}
\author{Chris Lee \& Christoph Klein}
\newcommand{\sphinxlogo}{}
\renewcommand{\releasename}{Release}
\makeindex

\makeatletter
\def\PYG@reset{\let\PYG@it=\relax \let\PYG@bf=\relax%
    \let\PYG@ul=\relax \let\PYG@tc=\relax%
    \let\PYG@bc=\relax \let\PYG@ff=\relax}
\def\PYG@tok#1{\csname PYG@tok@#1\endcsname}
\def\PYG@toks#1+{\ifx\relax#1\empty\else%
    \PYG@tok{#1}\expandafter\PYG@toks\fi}
\def\PYG@do#1{\PYG@bc{\PYG@tc{\PYG@ul{%
    \PYG@it{\PYG@bf{\PYG@ff{#1}}}}}}}
\def\PYG#1#2{\PYG@reset\PYG@toks#1+\relax+\PYG@do{#2}}

\def\PYG@tok@gd{\def\PYG@tc##1{\textcolor[rgb]{0.63,0.00,0.00}{##1}}}
\def\PYG@tok@gu{\let\PYG@bf=\textbf\def\PYG@tc##1{\textcolor[rgb]{0.50,0.00,0.50}{##1}}}
\def\PYG@tok@gt{\def\PYG@tc##1{\textcolor[rgb]{0.00,0.25,0.82}{##1}}}
\def\PYG@tok@gs{\let\PYG@bf=\textbf}
\def\PYG@tok@gr{\def\PYG@tc##1{\textcolor[rgb]{1.00,0.00,0.00}{##1}}}
\def\PYG@tok@cm{\let\PYG@it=\textit\def\PYG@tc##1{\textcolor[rgb]{0.25,0.50,0.56}{##1}}}
\def\PYG@tok@vg{\def\PYG@tc##1{\textcolor[rgb]{0.73,0.38,0.84}{##1}}}
\def\PYG@tok@m{\def\PYG@tc##1{\textcolor[rgb]{0.13,0.50,0.31}{##1}}}
\def\PYG@tok@mh{\def\PYG@tc##1{\textcolor[rgb]{0.13,0.50,0.31}{##1}}}
\def\PYG@tok@cs{\def\PYG@tc##1{\textcolor[rgb]{0.25,0.50,0.56}{##1}}\def\PYG@bc##1{\colorbox[rgb]{1.00,0.94,0.94}{##1}}}
\def\PYG@tok@ge{\let\PYG@it=\textit}
\def\PYG@tok@vc{\def\PYG@tc##1{\textcolor[rgb]{0.73,0.38,0.84}{##1}}}
\def\PYG@tok@il{\def\PYG@tc##1{\textcolor[rgb]{0.13,0.50,0.31}{##1}}}
\def\PYG@tok@go{\def\PYG@tc##1{\textcolor[rgb]{0.19,0.19,0.19}{##1}}}
\def\PYG@tok@cp{\def\PYG@tc##1{\textcolor[rgb]{0.00,0.44,0.13}{##1}}}
\def\PYG@tok@gi{\def\PYG@tc##1{\textcolor[rgb]{0.00,0.63,0.00}{##1}}}
\def\PYG@tok@gh{\let\PYG@bf=\textbf\def\PYG@tc##1{\textcolor[rgb]{0.00,0.00,0.50}{##1}}}
\def\PYG@tok@ni{\let\PYG@bf=\textbf\def\PYG@tc##1{\textcolor[rgb]{0.84,0.33,0.22}{##1}}}
\def\PYG@tok@nl{\let\PYG@bf=\textbf\def\PYG@tc##1{\textcolor[rgb]{0.00,0.13,0.44}{##1}}}
\def\PYG@tok@nn{\let\PYG@bf=\textbf\def\PYG@tc##1{\textcolor[rgb]{0.05,0.52,0.71}{##1}}}
\def\PYG@tok@no{\def\PYG@tc##1{\textcolor[rgb]{0.38,0.68,0.84}{##1}}}
\def\PYG@tok@na{\def\PYG@tc##1{\textcolor[rgb]{0.25,0.44,0.63}{##1}}}
\def\PYG@tok@nb{\def\PYG@tc##1{\textcolor[rgb]{0.00,0.44,0.13}{##1}}}
\def\PYG@tok@nc{\let\PYG@bf=\textbf\def\PYG@tc##1{\textcolor[rgb]{0.05,0.52,0.71}{##1}}}
\def\PYG@tok@nd{\let\PYG@bf=\textbf\def\PYG@tc##1{\textcolor[rgb]{0.33,0.33,0.33}{##1}}}
\def\PYG@tok@ne{\def\PYG@tc##1{\textcolor[rgb]{0.00,0.44,0.13}{##1}}}
\def\PYG@tok@nf{\def\PYG@tc##1{\textcolor[rgb]{0.02,0.16,0.49}{##1}}}
\def\PYG@tok@si{\let\PYG@it=\textit\def\PYG@tc##1{\textcolor[rgb]{0.44,0.63,0.82}{##1}}}
\def\PYG@tok@s2{\def\PYG@tc##1{\textcolor[rgb]{0.25,0.44,0.63}{##1}}}
\def\PYG@tok@vi{\def\PYG@tc##1{\textcolor[rgb]{0.73,0.38,0.84}{##1}}}
\def\PYG@tok@nt{\let\PYG@bf=\textbf\def\PYG@tc##1{\textcolor[rgb]{0.02,0.16,0.45}{##1}}}
\def\PYG@tok@nv{\def\PYG@tc##1{\textcolor[rgb]{0.73,0.38,0.84}{##1}}}
\def\PYG@tok@s1{\def\PYG@tc##1{\textcolor[rgb]{0.25,0.44,0.63}{##1}}}
\def\PYG@tok@gp{\let\PYG@bf=\textbf\def\PYG@tc##1{\textcolor[rgb]{0.78,0.36,0.04}{##1}}}
\def\PYG@tok@sh{\def\PYG@tc##1{\textcolor[rgb]{0.25,0.44,0.63}{##1}}}
\def\PYG@tok@ow{\let\PYG@bf=\textbf\def\PYG@tc##1{\textcolor[rgb]{0.00,0.44,0.13}{##1}}}
\def\PYG@tok@sx{\def\PYG@tc##1{\textcolor[rgb]{0.78,0.36,0.04}{##1}}}
\def\PYG@tok@bp{\def\PYG@tc##1{\textcolor[rgb]{0.00,0.44,0.13}{##1}}}
\def\PYG@tok@c1{\let\PYG@it=\textit\def\PYG@tc##1{\textcolor[rgb]{0.25,0.50,0.56}{##1}}}
\def\PYG@tok@kc{\let\PYG@bf=\textbf\def\PYG@tc##1{\textcolor[rgb]{0.00,0.44,0.13}{##1}}}
\def\PYG@tok@c{\let\PYG@it=\textit\def\PYG@tc##1{\textcolor[rgb]{0.25,0.50,0.56}{##1}}}
\def\PYG@tok@mf{\def\PYG@tc##1{\textcolor[rgb]{0.13,0.50,0.31}{##1}}}
\def\PYG@tok@err{\def\PYG@bc##1{\fcolorbox[rgb]{1.00,0.00,0.00}{1,1,1}{##1}}}
\def\PYG@tok@kd{\let\PYG@bf=\textbf\def\PYG@tc##1{\textcolor[rgb]{0.00,0.44,0.13}{##1}}}
\def\PYG@tok@ss{\def\PYG@tc##1{\textcolor[rgb]{0.32,0.47,0.09}{##1}}}
\def\PYG@tok@sr{\def\PYG@tc##1{\textcolor[rgb]{0.14,0.33,0.53}{##1}}}
\def\PYG@tok@mo{\def\PYG@tc##1{\textcolor[rgb]{0.13,0.50,0.31}{##1}}}
\def\PYG@tok@mi{\def\PYG@tc##1{\textcolor[rgb]{0.13,0.50,0.31}{##1}}}
\def\PYG@tok@kn{\let\PYG@bf=\textbf\def\PYG@tc##1{\textcolor[rgb]{0.00,0.44,0.13}{##1}}}
\def\PYG@tok@o{\def\PYG@tc##1{\textcolor[rgb]{0.40,0.40,0.40}{##1}}}
\def\PYG@tok@kr{\let\PYG@bf=\textbf\def\PYG@tc##1{\textcolor[rgb]{0.00,0.44,0.13}{##1}}}
\def\PYG@tok@s{\def\PYG@tc##1{\textcolor[rgb]{0.25,0.44,0.63}{##1}}}
\def\PYG@tok@kp{\def\PYG@tc##1{\textcolor[rgb]{0.00,0.44,0.13}{##1}}}
\def\PYG@tok@w{\def\PYG@tc##1{\textcolor[rgb]{0.73,0.73,0.73}{##1}}}
\def\PYG@tok@kt{\def\PYG@tc##1{\textcolor[rgb]{0.56,0.13,0.00}{##1}}}
\def\PYG@tok@sc{\def\PYG@tc##1{\textcolor[rgb]{0.25,0.44,0.63}{##1}}}
\def\PYG@tok@sb{\def\PYG@tc##1{\textcolor[rgb]{0.25,0.44,0.63}{##1}}}
\def\PYG@tok@k{\let\PYG@bf=\textbf\def\PYG@tc##1{\textcolor[rgb]{0.00,0.44,0.13}{##1}}}
\def\PYG@tok@se{\let\PYG@bf=\textbf\def\PYG@tc##1{\textcolor[rgb]{0.25,0.44,0.63}{##1}}}
\def\PYG@tok@sd{\let\PYG@it=\textit\def\PYG@tc##1{\textcolor[rgb]{0.25,0.44,0.63}{##1}}}

\def\PYGZbs{\char`\\}
\def\PYGZus{\char`\_}
\def\PYGZob{\char`\{}
\def\PYGZcb{\char`\}}
\def\PYGZca{\char`\^}
\def\PYGZsh{\char`\#}
\def\PYGZpc{\char`\%}
\def\PYGZdl{\char`\$}
\def\PYGZti{\char`\~}
% for compatibility with earlier versions
\def\PYGZat{@}
\def\PYGZlb{[}
\def\PYGZrb{]}
\makeatother

\begin{document}

\maketitle
\tableofcontents
\phantomsection\label{index::doc}


Contents:


\chapter{Example}
\label{testfiles/example::doc}\label{testfiles/example:example}\label{testfiles/example:welcome-to-ctools-s-documentation}
\begin{Verbatim}[commandchars=\\\{\}]
\PYG{g+gp}{\textgreater{}\textgreater{}\textgreater{} }\PYG{n}{Driver}\PYG{o}{.}\PYG{n}{initSystem}\PYG{p}{(}\PYG{l+s}{"}\PYG{l+s}{Solvated GMX}\PYG{l+s}{"}\PYG{p}{)}
\PYG{g+gp}{\textgreater{}\textgreater{}\textgreater{} }\PYG{c}{\PYGZsh{} Reading}
\PYG{g+gp}{\textgreater{}\textgreater{}\textgreater{} }\PYG{n}{Driver}\PYG{o}{.}\PYG{n}{loadctools}\PYG{p}{(}\PYG{l+s}{"}\PYG{l+s}{system\PYGZus{}GMX.top}\PYG{l+s}{"}\PYG{p}{)}
\PYG{g+gp}{\textgreater{}\textgreater{}\textgreater{} }\PYG{n}{Driver}\PYG{o}{.}\PYG{n}{loadStructure}\PYG{p}{(}\PYG{l+s}{"}\PYG{l+s}{system\PYGZus{}GMX.gro}\PYG{l+s}{"}\PYG{p}{)}
\PYG{g+gp}{\textgreater{}\textgreater{}\textgreater{} }\PYG{c}{\PYGZsh{} Writing}
\PYG{g+gp}{\textgreater{}\textgreater{}\textgreater{} }\PYG{n}{Driver}\PYG{o}{.}\PYG{n}{writeStructure}\PYG{p}{(}\PYG{l+s}{"}\PYG{l+s}{system\PYGZus{}GMX\PYGZus{}out.gro}\PYG{l+s}{"}\PYG{p}{)}
\PYG{g+gp}{\textgreater{}\textgreater{}\textgreater{} }\PYG{n}{Driver}\PYG{o}{.}\PYG{n}{writectools}\PYG{p}{(}\PYG{l+s}{"}\PYG{l+s}{system\PYGZus{}GMX\PYGZus{}out.top}\PYG{l+s}{"}\PYG{p}{)}
\end{Verbatim}


\chapter{System}
\label{system:module-ctools.System}\label{system::doc}\label{system:system}
\index{ctools.System (module)}
\index{System (class in ctools.System)}

\begin{fulllineitems}
\phantomsection\label{system:ctools.System.System}\pysiglinewithargsret{\strong{class }\code{ctools.System.}\bfcode{System}}{\emph{name=None}}{}
Bases: \code{object}

Initialize a new System object. This must be run before the system can be used.
\begin{description}
\item[{Args:}] \leavevmode
name (str): The name of the system

\end{description}

\begin{Verbatim}[commandchars=\\\{\}]
\PYG{g+gp}{\textgreater{}\textgreater{}\textgreater{} }\PYG{k}{print} \PYG{n}{\PYGZus{}\PYGZus{}init\PYGZus{}\PYGZus{}}\PYG{p}{(}\PYG{n}{name}\PYG{o}{=}\PYG{l+s}{'}\PYG{l+s}{sysname}\PYG{l+s}{'}\PYG{p}{)}
\end{Verbatim}

\index{addMolecule() (ctools.System.System method)}

\begin{fulllineitems}
\phantomsection\label{system:ctools.System.System.addMolecule}\pysiglinewithargsret{\bfcode{addMolecule}}{\emph{molecule}}{}
Append a molecule into the System.
\begin{description}
\item[{Args:}] \leavevmode
molecule (Molecule): The molecule object to be appended

\end{description}

\end{fulllineitems}


\index{getBoxVector() (ctools.System.System method)}

\begin{fulllineitems}
\phantomsection\label{system:ctools.System.System.getBoxVector}\pysiglinewithargsret{\bfcode{getBoxVector}}{}{}
Get the box vector coordinates

\end{fulllineitems}


\index{removeMoleculeType() (ctools.System.System method)}

\begin{fulllineitems}
\phantomsection\label{system:ctools.System.System.removeMoleculeType}\pysiglinewithargsret{\bfcode{removeMoleculeType}}{\emph{molecule}}{}
Remove a molecule from the System.
\begin{description}
\item[{Args:}] \leavevmode
molecule (Molecule): The molecule object to be removed

\end{description}

\end{fulllineitems}


\index{setBoxVector() (ctools.System.System method)}

\begin{fulllineitems}
\phantomsection\label{system:ctools.System.System.setBoxVector}\pysiglinewithargsret{\bfcode{setBoxVector}}{\emph{v1x}, \emph{v2x}, \emph{v3x}, \emph{v1y}, \emph{v2y}, \emph{v3y}, \emph{v1z}, \emph{v2z}, \emph{v3z}}{}
Sets the boxvector for the system. Assumes the box vector is in the correct form. {[}{[}v1x,v2x,v3x{]},{[}v1y,v2y,v3y{]},{[}v1z,v2z,v3z{]}{]}

\end{fulllineitems}


\end{fulllineitems}



\chapter{Molecule}
\label{molecule:molecule}\label{molecule::doc}\label{molecule:module-ctools.Molecule}
\index{ctools.Molecule (module)}
\index{Molecule (class in ctools.Molecule)}

\begin{fulllineitems}
\phantomsection\label{molecule:ctools.Molecule.Molecule}\pysiglinewithargsret{\strong{class }\code{ctools.Molecule.}\bfcode{Molecule}}{\emph{name=None}}{}
Bases: \code{object}

Initialize the molecule
\begin{description}
\item[{Args:}] \leavevmode
name (str): name of the molecule

\end{description}

\index{addAtom() (ctools.Molecule.Molecule method)}

\begin{fulllineitems}
\phantomsection\label{molecule:ctools.Molecule.Molecule.addAtom}\pysiglinewithargsret{\bfcode{addAtom}}{\emph{atom}}{}
Add and atom
\begin{description}
\item[{Args:}] \leavevmode
atom (atom): the atom to add into the molecule

\end{description}

\end{fulllineitems}


\index{getAtoms() (ctools.Molecule.Molecule method)}

\begin{fulllineitems}
\phantomsection\label{molecule:ctools.Molecule.Molecule.getAtoms}\pysiglinewithargsret{\bfcode{getAtoms}}{}{}
Return an orderedset of atoms

\end{fulllineitems}


\index{removeAtom() (ctools.Molecule.Molecule method)}

\begin{fulllineitems}
\phantomsection\label{molecule:ctools.Molecule.Molecule.removeAtom}\pysiglinewithargsret{\bfcode{removeAtom}}{\emph{atom}}{}
Remove Atom
\begin{description}
\item[{Args:}] \leavevmode
atom (atom): the atom to remove from the molecule

\end{description}

\end{fulllineitems}


\end{fulllineitems}



\chapter{MoleculeType}
\label{moleculeType:moleculetype}\label{moleculeType:module-ctools.MoleculeType}\label{moleculeType::doc}
\index{ctools.MoleculeType (module)}
\index{MoleculeType (class in ctools.MoleculeType)}

\begin{fulllineitems}
\phantomsection\label{moleculeType:ctools.MoleculeType.MoleculeType}\pysiglinewithargsret{\strong{class }\code{ctools.MoleculeType.}\bfcode{MoleculeType}}{\emph{name}}{}
Bases: \code{object}

Initialize the MoleculeType container
\begin{description}
\item[{Args:}] \leavevmode
name (str): the name of the moleculetype to add

\end{description}

\index{addForce() (ctools.MoleculeType.MoleculeType method)}

\begin{fulllineitems}
\phantomsection\label{moleculeType:ctools.MoleculeType.MoleculeType.addForce}\pysiglinewithargsret{\bfcode{addForce}}{\emph{force}}{}
Add a force to the moleculeType
\begin{description}
\item[{Args:}] \leavevmode
force (AbstractForce): Add a force or contraint to the moleculeType

\end{description}

\end{fulllineitems}


\index{addMolecule() (ctools.MoleculeType.MoleculeType method)}

\begin{fulllineitems}
\phantomsection\label{moleculeType:ctools.MoleculeType.MoleculeType.addMolecule}\pysiglinewithargsret{\bfcode{addMolecule}}{\emph{molecule}}{}
Add a molecule into the moleculetype container
\begin{description}
\item[{Args:   }] \leavevmode
molecule (Molecule): the molecule to append

\end{description}

\end{fulllineitems}


\index{getForce() (ctools.MoleculeType.MoleculeType method)}

\begin{fulllineitems}
\phantomsection\label{moleculeType:ctools.MoleculeType.MoleculeType.getForce}\pysiglinewithargsret{\bfcode{getForce}}{\emph{force}}{}
Get a force from the moleculeType
\begin{description}
\item[{Args:}] \leavevmode
force (AbstractForce): Retrieve a force from the moleculeType

\end{description}

\end{fulllineitems}


\index{getMolecule() (ctools.MoleculeType.MoleculeType method)}

\begin{fulllineitems}
\phantomsection\label{moleculeType:ctools.MoleculeType.MoleculeType.getMolecule}\pysiglinewithargsret{\bfcode{getMolecule}}{\emph{molecule}}{}
Get a molecule from the system
\begin{description}
\item[{Args:}] \leavevmode
molecule (Molecule): retrieve an equivalent molecule from the moleculetype

\end{description}

\end{fulllineitems}


\index{getNrexcl() (ctools.MoleculeType.MoleculeType method)}

\begin{fulllineitems}
\phantomsection\label{moleculeType:ctools.MoleculeType.MoleculeType.getNrexcl}\pysiglinewithargsret{\bfcode{getNrexcl}}{}{}
Gets the nrexcl

\end{fulllineitems}


\index{removeForce() (ctools.MoleculeType.MoleculeType method)}

\begin{fulllineitems}
\phantomsection\label{moleculeType:ctools.MoleculeType.MoleculeType.removeForce}\pysiglinewithargsret{\bfcode{removeForce}}{\emph{force}}{}
Remove a force from the moleculeType
\begin{description}
\item[{Args:}] \leavevmode
force (AbstractForce): Remove a force from the moleculeType

\end{description}

\end{fulllineitems}


\index{removeMolecule() (ctools.MoleculeType.MoleculeType method)}

\begin{fulllineitems}
\phantomsection\label{moleculeType:ctools.MoleculeType.MoleculeType.removeMolecule}\pysiglinewithargsret{\bfcode{removeMolecule}}{\emph{molecule}}{}
Remove a molecule from the system.
\begin{description}
\item[{Args:}] \leavevmode
molecule (Molecule): remove a molecule from the moleculeType

\end{description}

\end{fulllineitems}


\index{setNrexcl() (ctools.MoleculeType.MoleculeType method)}

\begin{fulllineitems}
\phantomsection\label{moleculeType:ctools.MoleculeType.MoleculeType.setNrexcl}\pysiglinewithargsret{\bfcode{setNrexcl}}{\emph{nrexcl}}{}
Set the nrexcl
\begin{description}
\item[{Args:}] \leavevmode
nrexcl (int): the value for nrexcl

\end{description}

\end{fulllineitems}


\end{fulllineitems}



\chapter{Atom}
\label{atom:module-ctools.Atom}\label{atom::doc}\label{atom:atom}
\index{ctools.Atom (module)}
\index{Atom (class in ctools.Atom)}

\begin{fulllineitems}
\phantomsection\label{atom:ctools.Atom.Atom}\pysiglinewithargsret{\strong{class }\code{ctools.Atom.}\bfcode{Atom}}{\emph{atomIndex}, \emph{atomName=None}, \emph{residueIndex=-1}, \emph{residueName=None}}{}
Bases: \code{object}

Create an Atom object
\begin{description}
\item[{Args:}] \leavevmode
atomIndex (int): index of atom in the molecule
atomName (str): name of the atom (eg., N, C, H, O) 
residueIndex (int): index of residue in the molecule
residueName (str): name of the residue (eg., THR, CYS)

\end{description}

\index{getAtomType() (ctools.Atom.Atom method)}

\begin{fulllineitems}
\phantomsection\label{atom:ctools.Atom.Atom.getAtomType}\pysiglinewithargsret{\bfcode{getAtomType}}{\emph{index=None}}{}
Gets the atomtype
\begin{description}
\item[{Args:}] \leavevmode
index (str): the value corresponding with type precedence (A Type, B Type)

\item[{Returns:}] \leavevmode
atomtype (list, str): Returns the atomtype list or the value at index if index is specified

\end{description}

\end{fulllineitems}


\index{getCgnr() (ctools.Atom.Atom method)}

\begin{fulllineitems}
\phantomsection\label{atom:ctools.Atom.Atom.getCgnr}\pysiglinewithargsret{\bfcode{getCgnr}}{\emph{index=None}}{}
Gets the Cgnr
\begin{description}
\item[{Args:}] \leavevmode
index (int): the index to retrieve, defaults to None

\item[{Returns:}] \leavevmode
cngr (dict, int): returns the index or the dictionary depending on if index is set

\end{description}

\end{fulllineitems}


\index{getCharge() (ctools.Atom.Atom method)}

\begin{fulllineitems}
\phantomsection\label{atom:ctools.Atom.Atom.getCharge}\pysiglinewithargsret{\bfcode{getCharge}}{\emph{index=None}}{}
Gets the charge of the atom
\begin{description}
\item[{Args:}] \leavevmode
index (int): index of the charge to retrieve defaults to None

\item[{Returns:}] \leavevmode
charge (float): Charge of the atom

\end{description}

\end{fulllineitems}


\index{getEpsilon() (ctools.Atom.Atom method)}

\begin{fulllineitems}
\phantomsection\label{atom:ctools.Atom.Atom.getEpsilon}\pysiglinewithargsret{\bfcode{getEpsilon}}{\emph{index=None}}{}
\end{fulllineitems}


\index{getForce() (ctools.Atom.Atom method)}

\begin{fulllineitems}
\phantomsection\label{atom:ctools.Atom.Atom.getForce}\pysiglinewithargsret{\bfcode{getForce}}{}{}
Gets the force of the atom
\begin{description}
\item[{Returns:}] \leavevmode
Tuple {[}fx, fy, fz{]}

\end{description}

\end{fulllineitems}


\index{getMass() (ctools.Atom.Atom method)}

\begin{fulllineitems}
\phantomsection\label{atom:ctools.Atom.Atom.getMass}\pysiglinewithargsret{\bfcode{getMass}}{\emph{index=None}}{}
Gets the mass of the atom
\begin{description}
\item[{Returns:}] \leavevmode
mass (float): mass of the atom
index (str): index to retrieve

\end{description}

\end{fulllineitems}


\index{getPosition() (ctools.Atom.Atom method)}

\begin{fulllineitems}
\phantomsection\label{atom:ctools.Atom.Atom.getPosition}\pysiglinewithargsret{\bfcode{getPosition}}{}{}
Gets the position fo the atom
\begin{description}
\item[{Returns:}] \leavevmode
Tuple {[}x, y, z{]}

\end{description}

\end{fulllineitems}


\index{getSigma() (ctools.Atom.Atom method)}

\begin{fulllineitems}
\phantomsection\label{atom:ctools.Atom.Atom.getSigma}\pysiglinewithargsret{\bfcode{getSigma}}{\emph{index=None}}{}
\end{fulllineitems}


\index{getVelocity() (ctools.Atom.Atom method)}

\begin{fulllineitems}
\phantomsection\label{atom:ctools.Atom.Atom.getVelocity}\pysiglinewithargsret{\bfcode{getVelocity}}{}{}
Gets the velocity of the atom
\begin{description}
\item[{Returns:}] \leavevmode
Tuple {[}vx, vy, vz{]}

\end{description}

\end{fulllineitems}


\index{setAtomType() (ctools.Atom.Atom method)}

\begin{fulllineitems}
\phantomsection\label{atom:ctools.Atom.Atom.setAtomType}\pysiglinewithargsret{\bfcode{setAtomType}}{\emph{index}, \emph{atomtype}}{}
Sets the atomtype
\begin{description}
\item[{Args:}] \leavevmode
atomtype (str): the atomtype of the atom
index (str): the value corresponding with type precedence (A Type, B Type)

\end{description}

\end{fulllineitems}


\index{setCgnr() (ctools.Atom.Atom method)}

\begin{fulllineitems}
\phantomsection\label{atom:ctools.Atom.Atom.setCgnr}\pysiglinewithargsret{\bfcode{setCgnr}}{\emph{index}, \emph{cgnr}}{}
Sets the Cgnr
\begin{description}
\item[{Args:}] \leavevmode
cgnr (int): The charge group number
index (int): the value corresponding with cgnr precedence

\end{description}

\end{fulllineitems}


\index{setCharge() (ctools.Atom.Atom method)}

\begin{fulllineitems}
\phantomsection\label{atom:ctools.Atom.Atom.setCharge}\pysiglinewithargsret{\bfcode{setCharge}}{\emph{index}, \emph{charge}}{}
Sets the charge of the atom
\begin{description}
\item[{Args:}] \leavevmode
charge (float): Charge of the atom
index (int): the index corresponding with charge precedence

\end{description}

\end{fulllineitems}


\index{setEpsilon() (ctools.Atom.Atom method)}

\begin{fulllineitems}
\phantomsection\label{atom:ctools.Atom.Atom.setEpsilon}\pysiglinewithargsret{\bfcode{setEpsilon}}{\emph{index}, \emph{epsilon}}{}
Sets the epsilon
\begin{description}
\item[{Args:}] \leavevmode
epsilon (float): epsilon of the atom
index(int): index corresponding to epsilon

\end{description}

\end{fulllineitems}


\index{setForce() (ctools.Atom.Atom method)}

\begin{fulllineitems}
\phantomsection\label{atom:ctools.Atom.Atom.setForce}\pysiglinewithargsret{\bfcode{setForce}}{\emph{fx}, \emph{fy}, \emph{fz}}{}
Sets the force of the atom
\begin{description}
\item[{Args:}] \leavevmode
fx (float): x force
fy (float): y force
fz (float): z force

\end{description}

\end{fulllineitems}


\index{setMass() (ctools.Atom.Atom method)}

\begin{fulllineitems}
\phantomsection\label{atom:ctools.Atom.Atom.setMass}\pysiglinewithargsret{\bfcode{setMass}}{\emph{index}, \emph{mass}}{}
Sets the mass of the atom
\begin{description}
\item[{Args:}] \leavevmode
mass (float): mass of the atom
index (str): the index corresponding with mass precedence (A Mass, B Mass)

\end{description}

\end{fulllineitems}


\index{setPosition() (ctools.Atom.Atom method)}

\begin{fulllineitems}
\phantomsection\label{atom:ctools.Atom.Atom.setPosition}\pysiglinewithargsret{\bfcode{setPosition}}{\emph{x}, \emph{y}, \emph{z}}{}
Sets the position of the atom
\begin{description}
\item[{Args:}] \leavevmode
x (float): x position
y (float): y position
z (float): z position

\end{description}

\end{fulllineitems}


\index{setSigma() (ctools.Atom.Atom method)}

\begin{fulllineitems}
\phantomsection\label{atom:ctools.Atom.Atom.setSigma}\pysiglinewithargsret{\bfcode{setSigma}}{\emph{index}, \emph{sigma}}{}
Sets the sigma
\begin{description}
\item[{Args:}] \leavevmode
sigma (float): sigma of the atom
index (int): index to insert at

\end{description}

\end{fulllineitems}


\index{setVelocity() (ctools.Atom.Atom method)}

\begin{fulllineitems}
\phantomsection\label{atom:ctools.Atom.Atom.setVelocity}\pysiglinewithargsret{\bfcode{setVelocity}}{\emph{vx}, \emph{vy}, \emph{vz}}{}
Sets the velocity of the atom
\begin{description}
\item[{Args:}] \leavevmode
vx (float): x velocity
vy (float): y velocity
vz (float): z velocity

\end{description}

\end{fulllineitems}


\end{fulllineitems}



\chapter{HashMap}
\label{hashMap:module-ctools.HashMap}\label{hashMap::doc}\label{hashMap:hashmap}
\index{ctools.HashMap (module)}
\index{HashMap (class in ctools.HashMap)}

\begin{fulllineitems}
\phantomsection\label{hashMap:ctools.HashMap.HashMap}\pysigline{\strong{class }\code{ctools.HashMap.}\bfcode{HashMap}}{}
Bases: \code{object}

Initializes the HashMap class

\index{add() (ctools.HashMap.HashMap method)}

\begin{fulllineitems}
\phantomsection\label{hashMap:ctools.HashMap.HashMap.add}\pysiglinewithargsret{\bfcode{add}}{\emph{key}}{}
Add a value to the container
\begin{description}
\item[{Args:}] \leavevmode
key: the key to add

\end{description}

\end{fulllineitems}


\index{get() (ctools.HashMap.HashMap method)}

\begin{fulllineitems}
\phantomsection\label{hashMap:ctools.HashMap.HashMap.get}\pysiglinewithargsret{\bfcode{get}}{\emph{key}}{}
Retrieve a key from the container
\begin{description}
\item[{Args:}] \leavevmode
key: key to retrieve

\end{description}

\end{fulllineitems}


\index{itervalues() (ctools.HashMap.HashMap method)}

\begin{fulllineitems}
\phantomsection\label{hashMap:ctools.HashMap.HashMap.itervalues}\pysiglinewithargsret{\bfcode{itervalues}}{}{}
Return a list of values

\end{fulllineitems}


\index{remove() (ctools.HashMap.HashMap method)}

\begin{fulllineitems}
\phantomsection\label{hashMap:ctools.HashMap.HashMap.remove}\pysiglinewithargsret{\bfcode{remove}}{\emph{key}}{}
Remove a key from the container
\begin{description}
\item[{Args:}] \leavevmode
key: key to remove

\end{description}

\end{fulllineitems}


\end{fulllineitems}



\chapter{OrderedSet}
\label{orderedSet:module-ctools.OrderedSet}\label{orderedSet::doc}\label{orderedSet:orderedset}
\index{ctools.OrderedSet (module)}
\index{OrderedSet (class in ctools.OrderedSet)}

\begin{fulllineitems}
\phantomsection\label{orderedSet:ctools.OrderedSet.OrderedSet}\pysigline{\strong{class }\code{ctools.OrderedSet.}\bfcode{OrderedSet}}{}
Bases: \code{object}

Initialize the orderedSet. Essentially a map coupled with a list.

\index{add() (ctools.OrderedSet.OrderedSet method)}

\begin{fulllineitems}
\phantomsection\label{orderedSet:ctools.OrderedSet.OrderedSet.add}\pysiglinewithargsret{\bfcode{add}}{\emph{key}}{}
Add a key to the container
\begin{description}
\item[{Args:}] \leavevmode
key: key to add

\end{description}

\end{fulllineitems}


\index{get() (ctools.OrderedSet.OrderedSet method)}

\begin{fulllineitems}
\phantomsection\label{orderedSet:ctools.OrderedSet.OrderedSet.get}\pysiglinewithargsret{\bfcode{get}}{\emph{key}}{}
Get a key from the container
\begin{description}
\item[{Args:}] \leavevmode
key: key to retrieve from the container

\end{description}

\end{fulllineitems}


\index{pop() (ctools.OrderedSet.OrderedSet method)}

\begin{fulllineitems}
\phantomsection\label{orderedSet:ctools.OrderedSet.OrderedSet.pop}\pysiglinewithargsret{\bfcode{pop}}{\emph{last=True}}{}
Pop a value from the container
\begin{description}
\item[{Args:}] \leavevmode
last (boolean): if true it pops the last element otherwise the first

\end{description}

\end{fulllineitems}


\index{remove() (ctools.OrderedSet.OrderedSet method)}

\begin{fulllineitems}
\phantomsection\label{orderedSet:ctools.OrderedSet.OrderedSet.remove}\pysiglinewithargsret{\bfcode{remove}}{\emph{key}}{}
Remove a value from the container
\begin{description}
\item[{Args:}] \leavevmode
key: key to remove:w

\end{description}

\end{fulllineitems}


\end{fulllineitems}



\chapter{Converter}
\label{converter:module-ctools.Converter}\label{converter::doc}\label{converter:converter}
\index{ctools.Converter (module)}\phantomsection\label{converter:module-ctools.Converter}
\index{ctools.Converter (module)}
\index{convert\_units() (in module ctools.Converter)}

\begin{fulllineitems}
\phantomsection\label{converter:ctools.Converter.convert_units}\pysiglinewithargsret{\code{ctools.Converter.}\bfcode{convert\_units}}{\emph{arg}, \emph{unit}}{}
Checks compatibility and converts units using simtk.units package
\begin{description}
\item[{Args:}] \leavevmode
arg (Quantity): quantity to be converted
unit (Unit): Unit to be converted to

\item[{Returns:}] \leavevmode
arg (Quantity): Quantity scaled to the new unit

\end{description}

\end{fulllineitems}



\chapter{Indices and tables}
\label{index:indices-and-tables}\begin{itemize}
\item {} 
\emph{genindex}

\item {} 
\emph{modindex}

\item {} 
\emph{search}

\end{itemize}


\renewcommand{\indexname}{Python Module Index}
\begin{theindex}
\def\bigletter#1{{\Large\sffamily#1}\nopagebreak\vspace{1mm}}
\bigletter{c}
\item {\texttt{ctools.Atom}}, \pageref{atom:module-ctools.Atom}
\item {\texttt{ctools.Converter}}, \pageref{converter:module-ctools.Converter}
\item {\texttt{ctools.HashMap}}, \pageref{hashMap:module-ctools.HashMap}
\item {\texttt{ctools.Molecule}}, \pageref{molecule:module-ctools.Molecule}
\item {\texttt{ctools.MoleculeType}}, \pageref{moleculeType:module-ctools.MoleculeType}
\item {\texttt{ctools.OrderedSet}}, \pageref{orderedSet:module-ctools.OrderedSet}
\item {\texttt{ctools.System}}, \pageref{system:module-ctools.System}
\end{theindex}

\renewcommand{\indexname}{Index}
\printindex
\end{document}
